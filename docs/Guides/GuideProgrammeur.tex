\documentclass{article}
\usepackage[utf8]{inputenc}
\usepackage[latin1]{inputenc}
\usepackage[francais]{babel}


\title{Le Guide du Programmeur}
\author{Nom Prenom }
\date{December 2015}

\usepackage{natbib}
\usepackage{graphicx}

\begin{document}

\maketitle

\section{Introduction}

There is a theory which states that if ever anyone discovers exactly what the Universe is for and why it is here, it will instantly disappear and be replaced by something even more bizarre and inexplicable.
There is another theory which states that this has already happened.

\begin{figure}[h!]
\centering
\includegraphics[scale=1.7]{universe.jpg}
\caption{The Universe}
\label{fig:univerise}
\end{figure}

\section{Architecture globale et Page d'accueil}

L'architecture du site web est découpée en trois grande partie

\begin{itemize}
\item[$\bullet$] La page d'accueil est dans le dossier 'src'
\item Les formulaires d'inscription sont dans 'src/inscription'
\item L'interface staff est dans 'src/staff'
\end{itemize}\\

Langages utilisés
\begin{itemize}
\item Côté Client: \begin{itemize}
        \item HTML
        \item CSS
        \item Javascript
    \end{itemize}
\item Côté Serveur: \begin{itemize}
        \item PHP
        \item MySQL
    \end{itemize}
\end{itemize}\\

Framework utilisés
\begin{itemize}
\item CSS: \begin{itemize}
        \item Bootstrap
        \item Font-Awesome
    \end{itemize}
\item JS: \begin{itemize}
        \item Bootstrap
        \item Datatable
        \item jquery
    \end{itemize}
\end{itemize}\\
La page d'accueil est directement située dans le dossier 'src' et correspond au fichier 'src/index.php'.

\fbox{Structure.png}


\section{Page d'accueil}

\begin{itemize}
\item Lien : src/index.php
\item Style : 'src/css'
\item JS : 'src/js'
\item Images 'src/images'
\end{itemize}

Le framework css utilisé bootstrap (bootstrap.min.css), et le template style.css\\
Le css personnalisés sont et monStyle.css\\
\fbox{CSS Accueil.png}
\\
Les frameworks javascript utilisés sont bootstrap, jquery, jqBootstrapValidation.js, html5shiv.js
et le template script.js\\
\fbox{JS Accueil.png}\\

Les fichiers contact\_me.js, contactStaff.js sont les fichiers permettant de faire fonctionner l'envoie de mail via la page d'accueil.\\
\fbox{mail accueil.png}

\section{Inscription Joueurs \& Propriétaire}
Les formulaires d'inscription pour les joueurs et les propriétaires sont situés dans le dossier 'src/inscription'. Ce sont les formulaires que les visiteurs utiliseront pour s'inscrire.\\
Le formulaire d'inscription de joueurs correspond au fichier 'src/inscription/index.php'.\\
Le formulaire d'inscription de joueurs correspond au fichier 'src/inscription/inscription-owner.php'.\\


\section{Interface Staff}

\subsection{Présentation}

\begin{itemize}
\item Lien : src/staff/pages
\item Librairies/Dépendances : src/staff/bower\_components
\item Style : 'src/staff/dist/css'
\item JS : 'src/staff/dist/js'
\item Outil génération PDF : 'src/staff/tcpdf'
\end{itemize}

Les librairies Bootstrap, font-awesome, et datatables ont été essentiellement utilisées.
Les css personnalisés sont dans dist/

\subsection{Pages}

Fichiers php dans le dossier src/staff/pages/ –> pages affichées dans l'application

Fichiers php dans le dossier src/staff/pages/php –> pages appelées par des methodes get/post pour faire du traitement de la base de donnée.

Fichiers inc dans le dossier src/staff/pages/php/inc -> fonctions pour avoir un code plus découpé

Le dossier 'src/staff/pages' contient tous les formulaires (ou vues) d'ajout et d'édition des entités (joueur, propriétaire, terrain, match, …). Ils sont représentés par des fichier PHP. Les formulaires d'édition sont caractérisés par le mot 'edit' en début de nom de fichier. Les formulaires d'ajout sont simplement désignés par le nom de l'entité qui leur correspond. Par exemple, le fichier 'court.php' correspond au formulaire d'ajout un terrain, et le fichier 'edit-court.php' correspond au formulaire d'édition d'un terrain.

Le dossier 'html' contient un fichier 'header.php' qui correspond à l'en-tête visible sur toutes les pages de l'interface Staff.

Le dossier 'php' contient toutes les fonctions d'interaction avec la base de données. Les fonctions d'ajout sont situées dans les fichiers dont le nom commence par 'add'. Les fonctions de suppression sont situées dans les fichiers dont le nom commence par 'delete'.
Le dossier 'inc' (dans 'php') contient les fonctions d'affichage et d'édition. Les fonctions d'édition sont situées dans les fichiers dont le nom commence par 'edit'. Les fonctions d'affichage sont situées dans les fichiers dont le nom commence par 'list' (pour les listes) et 'show' (pour les affichages unitaires).


\subsubsection{Vues}

Toutes les pages affichées se trouvent dans le dossier src/staff/pages/

Le dossier 'html' contient un fichier 'header.php' qui correspond à l'en-tête visible sur toutes les pages de l'interface Staff.

\fbox{header.png}


\fbox{liste.png}
\fbox{tableau.png}
\fbox{modal.png}
\fbox{form.png}

\begin{itemize}
\item{\textbf{Participant: }}
\begin{tabular}{ccc}
 Ajout & Edition & Affichage \\
 \hline
player.php & edit-player.php & list.php?=player\\
  &  & show.php?=player
\end{tabular}

\item{\textbf{Equipe: }}
\begin{tabular}{ccc}
 Ajout & Edition & Affichage \\
 \hline
team.php & edit-team.php & list-team.php\\
  & show-team.php
\end{tabular}

\item{\textbf{Match: }}
\begin{tabular}{ccc}
 Ajout & Edition & Affichage \\
 \hline
match.php & edit-match.php & list-match.php\\
  & & show-match.php
\end{tabular}

\item{\textbf{Catégorie: }}
\begin{tabular}{ccc}
 Ajout & Edition & Affichage \\
 \hline
category.php & edit-category.php & list.php?=category\\
  &  & show.php?=category
\end{tabular}

\item{\textbf{Propriétaire: }
\begin{tabular}{ccc}
 Ajout & Edition & Affichage \\
 \hline
owner.php & edit-owner.php & list.php?=owner\\
  &  & show.php?=owner
\end{tabular}

\item{\textbf{Terrain: }
\begin{tabular}{ccc}
 Ajout & Edition & Affichage \\
 \hline
court.php & edit-court.php & list.php?=court\\
  &  & show.php?=court
\end{tabular}

\item{\textbf{Staff: }
\begin{tabular}{ccc}
 Ajout & Edition & Affichage \\
 \hline
- & - & list.php?=court\\
  &  & show.php?=court
\end{tabular}

\item{\textbf{Extra: }}
\begin{tabular}{ccc}
 Ajout & Edition & Affichage \\
 \hline
extra.php & edit-extra.php & list-extras.php\\
  &  & show.php?=extra
\end{tabular}

\end{itemize}



\subsubsection{Modification BDD}
Toutes les modifications de la base de donnée affichées se trouvent dans le dossier src/staff/pages/php

\fbox{arborescence add.png}
\fbox{arborescence edit.png}


\begin{itemize}
\item{\textbf{Participant: }}
\begin{tabular}{ccc}
 Ajout & Edition & Suppresion \\
 \hline
add-new-pair.php & edit-player.php & delete-player.php\\
\end{tabular}

\item{\textbf{Equipe: }}
\begin{tabular}{ccc}
 Ajout & Edition & Suppresion \\
 \hline
add-new-team.php & edit-team.php & delete-team.php\\
\end{tabular}

\item{\textbf{Match: }}
\begin{tabular}{ccc}
 Ajout & Edition & Suppresion \\
 \hline
add-new-match.php & edit-match.php & delete-match.php\\
\end{tabular}

\item{\textbf{Catégorie: }}
\begin{tabular}{ccc}
 Ajout & Edition & Suppresion \\
 \hline
add-new-category.php & edit-category.php & delete-category.php\\
\end{tabular}

\item{\textbf{Propriétaire: }
\begin{tabular}{ccc}
 Ajout & Edition & Suppresion \\
 \hline
add-new-owner.php & edit-owner.php & delete-owner.php\\
\end{tabular}

\item{\textbf{Terrain: }
\begin{tabular}{ccc}
 Ajout & Edition & Suppresion \\
 \hline
add-new-court.php & edit-court.php & delete-court.php\\
\end{tabular}

\item{\textbf{Staff:}
\begin{tabular}{ccc}
 Ajout & Edition & Suppresion \\
 \hline
- & - & - \\
\end{tabular}

\item{\textbf{Extra: }
\begin{tabular}{ccc}
 Ajout & Edition & Suppresion \\
 \hline
add-new-extra.php & edit-extra.php & delete-extra.php\\
\end{tabular}

 \end{itemize}

\subsubsection{Fichiers .inc}
Les fichiers inc se trouvent dans le dossier src/staff/pages/php/inc
fonctions pour avoir un code plus découpé

\fbox{arborescence inc.png}

\section{Base de données}
La Base de données utilisées et en MySQL
\fbox{diagramme classe.png}

\section{Test}
\subsection{Selenium}

\subsection{Ghost Inspector}

\section{Conclusion}
``I always thought something was fundamentally wrong with the universe'' \citep{adams1995hitchhiker}

\bibliographystyle{plain}
\bibliography{references}

\end{document}
